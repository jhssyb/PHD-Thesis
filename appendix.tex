
\chapter{Functions and Datasets}

\section{Analytical functions}
\label{sec:functions}

Analytical functions---see~\cref{tab:functions}---with increasing numbers of input dimensions are presented, namely: \textit{(i)} \textit{Rosenbrock} ; \textit{(ii)} \textit{Michalewicz} ; \textit{(iii)} \textit{Branin} ; \textit{(iv)} \textit{Ishigami} ; and \textit{(v)} \textit{g-function}~\cite{molga2005,ishigami1990,Saltelli2007,Legratiet2016,forrester2008a}. They are all widely used because they are nonlinear and nonmonotonic. Moreover, two versions of the \emph{g-function} 11-D are also used. \textit{g-function (i)} 11-D demonstrates the behaviour of the methods whith a small number of input parameters contributing to the QoI, whereas \textit{g-function (ii)} 11-D exhibits more influent input parameters.

\begin{table}[h]
\centering
\setcellgapes{5pt}
\makegapedcells
\begin{tabular}{lll}
\toprule
Function&Hypercube & Definition \\
\cmidrule{1-3}
\textit{Rosenbrock} 2-D& $[-2.048, 2.048]^2$ & $
f(X_1, X_2) = \sum_{i = 1}^{d-1}[100(x_{i+1} - x_i^2)^2  +(x_i -1)^2]$\\
\textit{Michalewicz} 2-D& $[0, \pi]^2$ & $
f(X_1, X_2) = -\sum_{i=1}^d \sin(x_i)\sin^{2m}\left(\frac{ix_i^2}{\pi}\right)$\\
\textit{Branin} 2-D& $[-5, 10] \times [0, 15]$ &
$\begin{multlined}[t][0.6\linewidth]
f(X_1, X_2) = \left( x_2 - \frac{5.1}{4\pi^2}x_1^2 + \frac{5}{\pi}x_1 - 6
              \right)^2\\[-1ex]
+ 10 \left[ \left( 1 - \frac{1}{8\pi} \right) \cos(x_1) + 1 \right] + 5x_1\hfill
\end{multlined}$\\
\textit{Ishigami} 3-D& $[-\pi, \pi]^3$ & $ f(X_1, X_2, X_3) = \sin X_1 + 7 \sin^2 X_2 + 0.1 X_3^4 \sin X_1 $\\
\textit{g-function} 4-D& $[0, 1]^4$ & $
f(X_1, X_2, X_3, X_4) = \prod_{i=1}^4 \frac{\lvert 4X_i - 2\rvert + a_i}{1 + a_i}, \quad a_{i} = i$\\
\textit{g-function (i)} 11-D & $[0, 1]^{11}$ & 
$\begin{multlined}[t][0.6\linewidth]
f(X_1, ..., X_{11}) = \prod_{i=1}^{11} \frac{\lvert 4X_i - 2\rvert + a_i}{1 + a_i},\\[-1ex]
\hspace{50pt} \mathbf{a} = [1, 2, 5, 10, 20, 50, 100, 500, 1000, 1000, 1000]\hfill
\end{multlined}$\\
\textit{g-function (ii)} 11-D & $[0, 1]^{11}$ &
$\begin{multlined}[t][0.6\linewidth]
f(X_1, ..., X_{11}) = \prod_{i=1}^{11} \frac{\lvert 4X_i - 2\rvert + a_i}{1 + a_i},\\[-1ex]
\hspace{51pt} \mathbf{a} = [1, 2, 2, 3, 3, 10, 50, 50, 50, 100, 100]\hfill
\end{multlined}$\\
\bottomrule
\end{tabular}
\caption{Analytical functions considered sorted by increasing number of input parameters.}
\label{tab:functions}
\end{table}

\section{Datasets}
\label{sec:dataset}

\Cref{tab:dataset} presents two datasets. The first dataset (El Ni\~no) has no input-output relation and only features a temporal output. The second dataset Hydrodynamics features an input-output relation with spatially varying output. The datasets are as follows:
\begin{itemize}
\item The \emph{El Ni\~{n}o} dataset is a well-known functional dataset~\citep{Hyndman2009}. It consists in a time series of monthly averaged Sea Surface Temperature (SST) in degrees Celsius spatially averaged over the Pacific Ocean region (0-10°S and 90-80°W) from January 1950 to December 2007. The response variable is a vector of size 12 and the data set gathers 58 realizations. Data originate from NOAA ERSSTv5's database available at \href{http://www.cpc.ncep.noaa.gov/data/indices}{http://www.cpc.ncep.noaa.gov/data/indices}.
\item The \emph{Hydrodynamics} dataset gathers water levels (in m) computed with the 1-dimensional Shallow Water Equation MASCARET solver (opentelemac.org) for a 50~km reach of the Garonne river in South-West of France~\citep{Roy2017}. Uncertain inputs relate to 4 scalars: the friction coefficients of the river bed $Ks1, Ks2, Ks3$ defined over three homogeneous spatial areas, and the upstream boundary condition described by a constant scalar value for the inflow $Q$ in stationary flow. The response variable is a vector of size 463 (number of computation nodes for the 1D mesh) and an ensemble of 200 realizations is considered here. 
\end{itemize}

\begin{table}[!h]
\centering
\begin{tabular}{lccc}
\toprule
Dataset & Scalar input & Functional output & Sample size\\
\midrule % \cmidrule{2-3}
El Ni\~no & - & 12 & 58\\
Hydrodynamics & 4 & 463 & 200\\
\bottomrule
\end{tabular}
\caption{Description of the El Ni\~no and Hydrodynamics datasets.}
\label{tab:dataset}
\end{table}

