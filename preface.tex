% TODO acknowledgements
% TODO abstract
% TODO Résumé

\cleardoublepage
\section*{Acknowledgements}


%\newgeometry{top=1.5cm, bottom=1.75cm, inner=2.75cm, marginparwidth=2cm, textwidth=16cm, includeheadfoot}
%\newgeometry{a4paper}

\newpage
\section*{Abstract}

Uncertainties are predominant in the world that we know. Referring therefore to a nominal value is too restrictive, especially when it comes to complex systems. Understanding the nature and the impact of these uncertainties has become an important aspect of engineering work. On a societal point of view, uncertainties play a role in terms of decision-making. From the European Commission through the \emph{Better Regulation Guideline}, impact assessment are now advised to take uncertainties into account.

In order to understand the uncertainties, the mathematical field of Uncertainty Quantification (UQ) has been formed. UQ encompasses a large palette of statistical tools and it seeks to link a set of input perturbations on a system (design of experiments) towards a quantity of interest.

The purpose of this work is to propose improvements on various methodological aspects of UQ applied to costly numerical simulations. 

%Due to its cost, the numerical model is often replaced by a surrogate model which is cheaper to evaluate. A surrogate is a representation of the model which allows to compute a quantity of interest with respect to a design of experiment.

In this context, novel sampling and resampling approaches have been developed to better capture the variability when dealing with a high number of perturbed inputs. These allow to reduce the number of simulation required to describe the system. Moreover, novel methods are proposed to visualize uncertainties when dealing with either an high dimensional input parameter space or an high dimensional quantity of interest.

The developed methods developed can be used in various fields like hydraulic modelling and aerodynamic modelling. Their capabilities are demonstrated in realistic systems using well established Computational Fluid dynamics (CFD) tools. Lastly, they are not limited to the use of numerical experiments and can be used equally for real experiments.\\

\textbf{Keywords:} Uncertainty Quantification, Design of Experiments, High dimensions, Data visualization, Computational Fluid Dynamics %, Large Eddy Simulation

\newpage
\selectlanguage{french}
\section*{Résumé}

Les incertitudes font partie du monde qui nous entoure. Se limiter à une seule valeur nominale est bien souvent trop restrictif, et ce d'autant plus lorsqu'il est question de systèmes complexes. Comprendre la nature et l'impact de ces incertitudes est devenu un aspect important de tout travail d'ingénierie. D'un point de vue sociétal, les incertitudes jouent un rôle important dans les processus de décision. Les dernières recommandations de la Commission européenne en matière d'analyse de risque appuient sur l'importance du traitement des incertitudes.

Afin de comprendre les incertitudes, une nouvelle discipline mathématique appelée la quantification des incertitudes (UQ) a été créée. Ce domaine regroupe un large éventail de méthodes d'analyse statistique qui visent à lier des perturbations sur les paramètres d'entré sur un système (plan d'expérience) à une quantité d'intérêt.

L'objectif de ce travail de thèse est de proposer des améliorations sur divers aspects méthodologiques de la quantification des incertitudes dans le cadre de simulation numérique coûteuse.

Dans ce contexte, de nouvelles méthodes d'échantillonnage et de ré-échantillonnage ont été développées afin de mieux capturer la variabilité dans le cas d'un problème de grande dimension. Par ailleurs, de nouvelles méthodes de visualisation des incertitudes sont proposées dans le cas d'une grande dimension des paramètres d'entrée et d'une grande dimension de la quantité d'intérêt.

Les méthodes développées peuvent être utilisées dans divers domaines comme la modélisation hydraulique ou encore la modélisation aérodynamique. Leur apport est démontré sur des systèmes réalistes en faisant appel à des outils de mécanique des fluides numérique. Enfin, ces méthodes ne sont pas seulement utilisables dans le cadre de simulation numérique, mais elles peuvent être utilisées sur de réel dispositif expérimentaux.\\

\textbf{Mots clés :} Quantification d'incertitudes, Plan d'expérience, Grande dimension, Visualisation de données, Mécanique des fluides numérique %, Large Eddy Simulation

\selectlanguage{english}

%\restoregeometry



